\documentclass[]{article}
\usepackage{lmodern}
\usepackage{amssymb,amsmath}
\usepackage{ifxetex,ifluatex}
\usepackage{fixltx2e} % provides \textsubscript
\ifnum 0\ifxetex 1\fi\ifluatex 1\fi=0 % if pdftex
  \usepackage[T1]{fontenc}
  \usepackage[utf8]{inputenc}
\else % if luatex or xelatex
  \ifxetex
    \usepackage{mathspec}
    \usepackage{xltxtra,xunicode}
  \else
    \usepackage{fontspec}
  \fi
  \defaultfontfeatures{Mapping=tex-text,Scale=MatchLowercase}
  \newcommand{\euro}{€}
\fi
% use upquote if available, for straight quotes in verbatim environments
\IfFileExists{upquote.sty}{\usepackage{upquote}}{}
% use microtype if available
\IfFileExists{microtype.sty}{%
\usepackage{microtype}
\UseMicrotypeSet[protrusion]{basicmath} % disable protrusion for tt fonts
}{}
\usepackage[margin=1in]{geometry}
\usepackage{longtable,booktabs}
\ifxetex
  \usepackage[setpagesize=false, % page size defined by xetex
              unicode=false, % unicode breaks when used with xetex
              xetex]{hyperref}
\else
  \usepackage[unicode=true]{hyperref}
\fi
\hypersetup{breaklinks=true,
            bookmarks=true,
            pdfauthor={Report prepared by Cody Flagg and Katie Jones},
            pdftitle={Site Characterization Data Quality Report},
            colorlinks=true,
            citecolor=blue,
            urlcolor=blue,
            linkcolor=magenta,
            pdfborder={0 0 0}}
\urlstyle{same}  % don't use monospace font for urls
\setlength{\parindent}{0pt}
\setlength{\parskip}{6pt plus 2pt minus 1pt}
\setlength{\emergencystretch}{3em}  % prevent overfull lines
\setcounter{secnumdepth}{5}

%%% Use protect on footnotes to avoid problems with footnotes in titles
\let\rmarkdownfootnote\footnote%
\def\footnote{\protect\rmarkdownfootnote}

%%% Change title format to be more compact
\usepackage{titling}
\setlength{\droptitle}{-2em}
  \title{Site Characterization Data Quality Report}
  \pretitle{\vspace{\droptitle}\centering\huge}
  \posttitle{\par}
  \author{Report prepared by Cody Flagg and Katie Jones}
  \preauthor{\centering\large\emph}
  \postauthor{\par}
  \date{}
  \predate{}\postdate{}




\begin{document}

\maketitle


{
\hypersetup{linkcolor=black}
\setcounter{tocdepth}{2}
\tableofcontents
}
\section{Overview}\label{overview}

This data quality report is based on data files received from ARCADIS on
2/3/2015 forthe five sites at which vegetation characterization field
work occurred (over the period November 10th - January 21st). The
quality checks contained in this report represent the extent of what we
are capable of assessing without field checking contractor performance.
An additional quality report may still be filed for field work. For the
purpose of this summary, we have divided our concerns into two general
areas - issues related to \textbf{species identification} and issues
related to \textbf{data entry}:

\textbf{Species ID} - Two issues arose with respect to species
identifications:

\begin{enumerate}
\def\labelenumi{\arabic{enumi}.}
\item
  Incomplete identification - At three sites, HARV, SCBI and TALL, only
  a portion of the dataset includes values in the
  ``\texttt{scientificName}'' field (29\% complete at HARV, 67\%
  complete at SCBI and 16\% complete at TALL). Field work at all of
  these sites was initiated following fall senescence. In recognition of
  increased difficulty in positively identifying individuals to species,
  we relaxed the acceptable resolution to genus level at HARV and SCBI
  (10/30/14 Git issue \#3). This was reiterated in our 11/31/14
  conference call as species ID should be to ``\emph{Finest resolution
  possible given the conditions}'' (see meeting notes posted on GitHub).
  \textbf{In no instance did we agree that identification was not
  necessary}.
\item
  Use of incorrect data type in \texttt{scientificName} field - NEON
  uses USDA Plants as the taxonomic authority for naming/coding
  conventions but the data entry field for species is
  \texttt{scientificName}, which is defined in the Access DB
  \texttt{vstFieldSummary} table as `Binomial latin name', this is
  reinforced through the constrained set of values available as a
  dropdown list in the \texttt{scientificName} field in the
  \texttt{vst\_perindividual\_in} table (See guidance in Data
  Transcription protocol section 8.3.2.e.iii).
\end{enumerate}

Nonetheless, all \texttt{scientificName} entries in the HARV and JERC
datasets are codes. For JERC, converting from USDA code to
\texttt{scientificName} should be simple to update as all codes appear
to be valid USDA Plants accepted codes. On the other hand, in the HARV
dataset, most of the codes used \emph{are not} USDA codes, we have no
way to confidently assign a \texttt{scientificName} to most of these
records. See the print out of unique values in the
\texttt{scientificName} field (section 3.1, below) to identify
additional errors; \textbf{most of these are avoidable by utilizing the
drop down list in the Access DB}.

\textbf{Data entry} - Some of these errors may be attributable to data
transcription, others may be errors that originate in field data
collection. In either case, data with these errors cannot be ingested
into the NEON database.

\begin{itemize}
\itemsep1pt\parskip0pt\parsep0pt
\item
  \texttt{Numeric values in remarks field}. \textbf{Are these values
  data?} Remarks are not machine readable, as such, we are unable to
  make use of essential information if it is included in this field.
\item
  \texttt{Invalid pointIDs}, NEON cannot generate georeferenced stem
  maps if stems are mapped from points other than those identified in
  the protocol
\item
  \texttt{Invalid growthForms}, only growthForms listed in the protocol
  will be recognized by our ingest algorithms.\\
\item
  \texttt{Invalid Azimuth values}, individuals are plotted as being
  outside of the designated plot boundaries.
\end{itemize}

\textbf{See site specific summaries below and supplemental error reports
(in folder ``\texttt{QA\_Files}'') for more details.}

\section{Plot Completion}\label{plot-completion}

\begin{itemize}
\itemsep1pt\parskip0pt\parsep0pt
\item
  Provides a summary for which data are reported at each site \#\# Plot
  Count
\item
  Quick check: How many plots were visited?
\item
  HARV, SCBI and JERC have blank entries in field: \texttt{plotID}.
\end{itemize}

\begin{longtable}[c]{@{}lr@{}}
\caption{Plot Count by Site}\tabularnewline
\toprule
Site & PlotCount\tabularnewline
\midrule
\endfirsthead
\toprule
Site & PlotCount\tabularnewline
\midrule
\endhead
HARV & 18\tabularnewline
SCBI & 20\tabularnewline
JERC & 20\tabularnewline
OSBS & 15\tabularnewline
TALL & 20\tabularnewline
\bottomrule
\end{longtable}

\subsection{Missing Plots}\label{missing-plots}

\begin{itemize}
\itemsep1pt\parskip0pt\parsep0pt
\item
  These plots are in the provided NEON Excel files for each site, but
  were not found in the data files received from Arcadis.
\item
  Git issue \#9 documents that NEON Field Ops completed
  \texttt{HARV\_036 and HARV\_050}, please verify if this is also the
  case for missing plots at OSBS. SCBI appears to be a data entry error,
  please fix.
\end{itemize}

\begin{verbatim}
## [1] "HARV_050" "HARV_036" "OSBS_025" "OSBS_036" "OSBS_042" "OSBS_043"
## [7] "OSBS_044"
\end{verbatim}

\subsection{Extra Plots}\label{extra-plots}

\begin{itemize}
\itemsep1pt\parskip0pt\parsep0pt
\item
  This lists plots that were found in the received files from Arcadis
  that are not included in the lists provided by NEON.
\item
  This appears to be a data entry error.
\end{itemize}

\begin{verbatim}
## character(0)
\end{verbatim}

\section{Duplicate and Unique Tags}\label{duplicate-and-unique-tags}

\begin{itemize}
\itemsep1pt\parskip0pt\parsep0pt
\item
  This summarizes the number of duplicate \texttt{tagID} entries per
  site.
\item
  See the supplemental .csv files (File suffix:
  \texttt{\_duplicateTags}) for lists of specific duplicate tag IDs by
  site.
\item
  The first row in this table (with a blank \texttt{siteID}) indicates
  that 8 entries are missing a \texttt{plotID}.
\end{itemize}

\begin{longtable}[c]{@{}lr@{}}
\caption{Duplicate Tags by Site}\tabularnewline
\toprule
siteID & Count\tabularnewline
\midrule
\endfirsthead
\toprule
siteID & Count\tabularnewline
\midrule
\endhead
& 4\tabularnewline
HARV & 120\tabularnewline
JERC & 3\tabularnewline
OSBS & 9\tabularnewline
SCBI & 1\tabularnewline
\bottomrule
\end{longtable}

\subsection{Unique species list by site
(\texttt{scientificName})}\label{unique-species-list-by-site-scientificname}

\begin{itemize}
\itemsep1pt\parskip0pt\parsep0pt
\item
  Some unique records may be attributable to spelling and capitalization
  errors during data transcription
\end{itemize}

\begin{verbatim}
## [1] "HARV"
## 
## *  
## * Betula 
## * Fagus grandifolia 
## * Kalmia latifolia 
## * Pinus 
## * Quercus 
## * Tsuga canadensis 
## 
## <!-- end of list -->
## 
## [1] "SCBI"
## 
## *  
## * Acer 
## * Acer negundo 
## * Acer rubrum 
## * Berberis 
## * Berberis thunbergii 
## * Carya 
## * Carya tomentosa 
## * Celtis occidentalis 
## * Cornus florida 
## * Diospyros virginiana 
## * fraxinus 
## * Fraxinus 
## * Fraxinus americana 
## * Fraxinus americana var. microcarpa 
## * Fraxinus pennsylvanica 
## * Juglans nigra 
## * Lindera benzoin 
## * Liquidambar styraciflua 
## * Liriodendron tulipifera 
## * Platanus occidentalis 
## * Prunus serotina 
## * Quercus 
## * Quercus alba 
## * Quercus rubra 
## * Rosa multiflora 
## * Rubus 
## * Rubus phoedalibarda 
## * Rubus phoenicolasius 
## * Sassafras albidum 
## * Sassafras albidum var. molle 
## * Symphoricarpos orbiculatus 
## * Ulmus 
## * Ulmus americana 
## * Vitis 
## 
## <!-- end of list -->
## 
## [1] "JERC"
## 
## *  
## * Acer floridanum 
## * Aesculus pavia 
## * Bignonia capreolata 
## * Bignonia capreolataBignonia capreolata 
## * Campsis radicans 
## * Carya glabra 
## * Celtis laevigata 
## * Celtis occidentalis 
## * Cercis canadensis 
## * Cocculus carolinus 
## * Cornus florida 
## * Crataegus flava 
## * Crataegus flava Aiton 
## * Crataegus spathulata 
## * Diospyros virginiana 
## * Fraxinus americana 
## * Gelsemium sempervirens 
## * Ilex glabra 
## * Ilex vomitoria 
## * Juniperus virginiana 
## * Liquidambar styraciflua 
## * Malus angustifolia 
## * Morus rubra 
## * Nyssa sylvatica 
## * Parthenocissus quinquefolia 
## * Pinus clausa 
## * Pinus echinata 
## * Pinus palustris 
## * Prunus americana 
## * Prunus serotina 
## * Quercus falcata 
## * Quercus geminata 
## * Quercus hemisphaerica 
## * Quercus hemispherica 
## * Quercus incana 
## * Quercus laevis 
## * Quercus lyrata 
## * Quercus margaretta 
## * Quercus minima 
## * Quercus nigra 
## * Quercus virginiana 
## * Rhus copallinum 
## * Sassafras albidum 
## * Sideroxylon languinosum 
## * Sideroxylon lanuginosum 
## * Smilax bona-nox 
## * Smilax rotundifolia 
## * Smilax smallii 
## * Taxodium distichum 
## * Tilia americana 
## * Toxicodendron radicans 
## * Ulmus alata 
## * Unknown 
## * Vaccinium arboreum 
## * VIBURNUM 
## * Viburnum prunifolium 
## * Viburnum sp. 
## * Vitis rotundifola 
## * Vitis rotundifolia 
## 
## <!-- end of list -->
## 
## [1] "OSBS"
## 
## *  
## * Asimina incana 
## * Diospyros virginiana 
## * Pinus palustris 
## * Quercus geminata 
## * Quercus geminata Small 
## * Quercus laevis 
## * Quercus margarettae 
## * Vaccinium arboreum 
## 
## <!-- end of list -->
## 
## [1] "TALL"
## 
## *  
## * Acer rubrum 
## * Carya 
## * Cornus 
## * Ilex opaca 
## * Liquidambar styraciflua 
## * Liriodendron tulipifera 
## * Pinus 
## * Pinus palustris 
## * Pinus taeda 
## * Quercus 
## * Quercus alba 
## * Rhus 
## * Symplocos tinctoria 
## * Vaccinium arboreum 
## 
## <!-- end of list -->
\end{verbatim}

\subsection{Unique growth forms list by
site}\label{unique-growth-forms-list-by-site}

\begin{itemize}
\itemsep1pt\parskip0pt\parsep0pt
\item
  \textbf{Acceptable values} : lia, sbt, mbt, sms, sis, smt, sap
\end{itemize}

\begin{verbatim}
## $HARV
## [1] ""    "mbt" "sap" "sbt" "sis" "smt"
## 
## $SCBI
## [1] ""    "lia" "mbt" "sap" "sbt" "sis" "sms" "smt"
## 
## $JERC
## [1] "lia" "mbt" "sap" "sbt" "sis" "sms" "smt"
## 
## $OSBS
## [1] ""    "mbt" "sap" "sbt" "sis" "sms" "smt"
## 
## $TALL
## [1] "lia" "mbt" "sap" "sbt" "sis" "sms" "smt" "snt"
\end{verbatim}

\section{Point and Azimuth
Validation}\label{point-and-azimuth-validation}

\begin{itemize}
\itemsep1pt\parskip0pt\parsep0pt
\item
  This checks that all pointIDs are valid for plot dimensions and
  azimuth values are within pre-defined ranges for specific pointID.
\item
  See supplemental .csv files (file suffix:
  \texttt{pointID\_azimuth\_QF}) for specific details regarding where
  \texttt{pointID} and \texttt{stemAzimuth} errors occur.\\
\item
  ``\texttt{pointIDQF}'' and ``\texttt{azimuthQF}'' fields flag errors
  for each row in the supplemental files; ``\textbf{1}'' = no error,
  ``\textbf{-9999}'' = potential error.
\end{itemize}

\begin{longtable}[c]{@{}llrr@{}}
\caption{Point/Azimuth Errors by Site}\tabularnewline
\toprule
& siteID & pointID.error & azimuth.error\tabularnewline
\midrule
\endfirsthead
\toprule
& siteID & pointID.error & azimuth.error\tabularnewline
\midrule
\endhead
2 & HARV & 0 & 3\tabularnewline
3 & JERC & 0 & 0\tabularnewline
4 & OSBS & 0 & 9\tabularnewline
5 & SCBI & 0 & 6\tabularnewline
6 & TALL & 0 & 8\tabularnewline
\bottomrule
\end{longtable}

\pagebreak

\subsection{Growth Form Measurements by
Site}\label{growth-form-measurements-by-site}

\begin{itemize}
\itemsep1pt\parskip0pt\parsep0pt
\item
  This table summarizes the types of growth forms that have stem
  distance and azimuth measurements.
\item
  Note that a blank in the ``growthForm'' column suggests there is a
  missing entry.
\end{itemize}

\begin{longtable}[c]{@{}llr@{}}
\caption{Growth Form Measurements by Site}\tabularnewline
\toprule
siteID & growthForm & Count\tabularnewline
\midrule
\endfirsthead
\toprule
siteID & growthForm & Count\tabularnewline
\midrule
\endhead
& & 5\tabularnewline
HARV & & 1\tabularnewline
HARV & mbt & 562\tabularnewline
HARV & sap & 352\tabularnewline
HARV & sbt & 2213\tabularnewline
HARV & sis & 8\tabularnewline
HARV & smt & 1339\tabularnewline
JERC & lia & 160\tabularnewline
JERC & mbt & 23\tabularnewline
JERC & sap & 1096\tabularnewline
JERC & sbt & 455\tabularnewline
JERC & sis & 59\tabularnewline
JERC & sms & 379\tabularnewline
JERC & smt & 347\tabularnewline
OSBS & & 3\tabularnewline
OSBS & mbt & 14\tabularnewline
OSBS & sap & 113\tabularnewline
OSBS & sbt & 659\tabularnewline
OSBS & sis & 4\tabularnewline
OSBS & sms & 373\tabularnewline
OSBS & smt & 318\tabularnewline
SCBI & & 1\tabularnewline
SCBI & lia & 63\tabularnewline
SCBI & mbt & 53\tabularnewline
SCBI & sap & 488\tabularnewline
SCBI & sbt & 930\tabularnewline
SCBI & sis & 512\tabularnewline
SCBI & sms & 288\tabularnewline
SCBI & smt & 290\tabularnewline
TALL & lia & 5\tabularnewline
TALL & mbt & 21\tabularnewline
TALL & sap & 544\tabularnewline
TALL & sbt & 732\tabularnewline
TALL & sis & 93\tabularnewline
TALL & sms & 111\tabularnewline
TALL & smt & 276\tabularnewline
TALL & snt & 1\tabularnewline
\bottomrule
\end{longtable}

\pagebreak

\section{Missing Scientific Names}\label{missing-scientific-names}

\begin{itemize}
\itemsep1pt\parskip0pt\parsep0pt
\item
  This table summarizes how many scientific names are missing
  (i.e.~blanks, NA, unknown, unk etc.).
\end{itemize}

\begin{longtable}[c]{@{}lrrrr@{}}
\caption{Missing Scientific Names by Site}\tabularnewline
\toprule
siteID & NA & Blank & Unknown & Blank\%\tabularnewline
\midrule
\endfirsthead
\toprule
siteID & NA & Blank & Unknown & Blank\%\tabularnewline
\midrule
\endhead
HARV & 0 & 3425 & 0 & 76.468\tabularnewline
SCBI & 0 & 851 & 0 & 32.419\tabularnewline
JERC & 0 & 1 & 1 & 0.040\tabularnewline
OSBS & 0 & 2 & 0 & 0.135\tabularnewline
TALL & 0 & 1499 & 0 & 84.072\tabularnewline
\bottomrule
\end{longtable}

\section{Tagged Plants by Site}\label{tagged-plants-by-site}

\begin{itemize}
\itemsep1pt\parskip0pt\parsep0pt
\item
  This table summarizes the number of tags per plot per site.
\end{itemize}

\begin{longtable}[c]{@{}lrrr@{}}
\caption{Tags Used by Site}\tabularnewline
\toprule
siteID & minTags & maxtags & meanTags\tabularnewline
\midrule
\endfirsthead
\toprule
siteID & minTags & maxtags & meanTags\tabularnewline
\midrule
\endhead
HARV & 4 & 380 & 236\tabularnewline
SCBI & 54 & 247 & 131\tabularnewline
JERC & 48 & 331 & 126\tabularnewline
OSBS & 1 & 145 & 93\tabularnewline
TALL & 36 & 151 & 89\tabularnewline
\bottomrule
\end{longtable}

\end{document}
